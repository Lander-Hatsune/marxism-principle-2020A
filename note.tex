% Created 2020-11-06 五 13:36
% Intended LaTeX compiler: pdflatex
\documentclass[11pt]{article}
\usepackage[utf8]{inputenc}
\usepackage[T1]{fontenc}
\usepackage{graphicx}
\usepackage{grffile}
\usepackage{longtable}
\usepackage{wrapfig}
\usepackage{rotating}
\usepackage[normalem]{ulem}
\usepackage{amsmath}
\usepackage{textcomp}
\usepackage{amssymb}
\usepackage{capt-of}
\usepackage{hyperref}
\usepackage{minted}
\author{LanderX}
\date{}
\title{马克思主义原理}
\hypersetup{
 pdfauthor={LanderX},
 pdftitle={马克思主义原理},
 pdfkeywords={},
 pdfsubject={},
 pdfcreator={Emacs 27.1 (Org mode 9.4)}, 
 pdflang={English}}
\begin{document}

\maketitle
\tableofcontents


\section{导论: 马克思主义}
\label{sec:orgfe277d8}

\begin{quote}
一源万派
\end{quote}

\begin{itemize}
\item 学习的基本原则: 追本溯源
\end{itemize}

\subsection{一源: 马克思}
\label{sec:org538aea5}
\subsubsection{年龄分段}
\label{sec:org875625f}
不同时期观点不一样, 不要混淆

\begin{itemize}
\item 1818 - 1845 青年时期 
\begin{itemize}
\item 公知, 反对专制
\item 不强调科学必然, 抽象的人道主义
\item 西方中心论
\end{itemize}
\item 1845 - 1871 成熟时期
\begin{itemize}
\item \textbf{1845: 《德意志形态》, 《关于费尔巴哈的提纲》}
\item 强调科学必然规律 (毛:"历史规律不可抗拒, 美帝国主义必然灭亡")
\item 同样具有人道主义, 科学的人道主义
\item 受环境影响, 对东方仍具有偏见, 称古代中国"木乃伊"(朝代更替循环, 死而不衰), "活化石", "野蛮人", "半野蛮人"
\item 关注中国, 包括近代的革命
\item 东方社会也要经历西方的道路
\end{itemize}
\item 1871 - 1883 晚年
\begin{itemize}
\item \textbf{1871: 法兰西内战和巴黎公社}
\item 东方社会具备意志性, 有可能不经过资本主义的卡夫丁峡谷, 直接进入社会主义
\end{itemize}
\end{itemize}

\subsubsection{马克思和恩格斯: 马恩差异论}
\label{sec:orgee5dc6c}

\begin{itemize}
\item 哲学理论有些理解不同
\item 立场并不对立
\item 还存在对立论, 一致论
\item 梅林: 马克思首先是一名斗士, 在有斗争的机会时, 马克思会参与斗争
\end{itemize}

\subsection{万派}
\label{sec:org94a3791}
\subsubsection{第二国际}
\label{sec:orge6c9234}
\begin{itemize}
\item 第一国际的影子: 国际歌, 巴黎公社, 鲍迪埃. 共产党会议的ED
\item 1889, 社会民主工人建立第二国际, 恩格斯领导, 调和各党派
\item 恩格斯去世后, 第二国际分化斗争
\begin{itemize}
\item 二战背景下, 国际主义和民族主义的斗争
\begin{itemize}
\item 左派(革命派): 坚持马克思国际主义, 考虑战争是正义的吗? 开战者为了利益胁迫了整个民族. 不参与战争, 共同保卫和平.
\item 右派: 民族主义, 为本国而战. 大部分政党滑向了民族主义, 支持"防御性战争"
\end{itemize}
\item 革命主义和改良主义的斗争
\begin{itemize}
\item 左派: 马克思: 暴力革命方式消灭资产阶级
\begin{itemize}
\item 列宁: 让自己的祖国在战争中失败, 转国际战争为国内战争, 战争引发革命(毛: 二十世纪的主题是战争与革命)
\end{itemize}
\item 右派: 改良(修正)主义: 议会方式获取社会主导权
\item 中派: 考斯基, "共产党人的教皇", 调和左右派, 维持第二国际形式的团结. 
\begin{itemize}
\item 战争开始后导向右派
\end{itemize}
\end{itemize}
\end{itemize}
\item 第二国际演化为现代欧洲**民主社会主义**
\begin{itemize}
\item \textbf{不要消灭私有制}, 搞国家调控干预, 福利制度, 减小贫富差距 -> 社会主义特点
\item 瑞典等北欧国家
\item \textbf{指导思想多元化}, 以马克思主义, 西方人道主义, 近代启蒙思想等均为指导思想
\item \textbf{不采取暴力革命}, 采取**议会斗争**
\item 区分民主社会党, 社会党, 民主党, 工人党, 共产党
\begin{itemize}
\item 法国共产党 -> 社会民主党
\item 日本共产党 -> 社会民主党
\item \textbf{叫不叫共产党和是不是共产党无关}
\end{itemize}
\item 建立社会党国际
\end{itemize}
\end{itemize}

\subsubsection{列宁: 第三国际}
\label{sec:org85c9945}
\begin{itemize}
\item 第二国际纷纷倾向右派, 列宁认为第二国际破产, 建立第三国际继承马克思思想
\item 进入第三国际尽量更名共产党
\item \textbf{正统的马克思主义}: 中国共产党在其中
\item 列宁去世后, 第三国际在苏联内分化
\begin{itemize}
\item 斯大林 -> 胜出
\item 托洛茨基
\begin{itemize}
\item 两人对立: 一国不能建成社会主义
(马克思: 全世界爆发革命, 才可能进入共产主义, 俄国可能率先爆发, 但是俄国的革命只有引发西欧革命并互为补充, 才有可能胜利)
列宁对此一直很犹豫
\item 被斯大林驱逐出境暗杀
\end{itemize}
\item 布哈林
\begin{itemize}
\item 两人对立: 长期利用新经济政策
列宁: 新经济政策, 一定范围允许私有, 但不知应持续多久
\item 被审判枪决
\item 戈尔巴乔夫为之平反
\end{itemize}
\end{itemize}
\item 马克思主义出口各国
\begin{itemize}
\item 中国: 斯大林->毛->中国特色
\end{itemize}
\end{itemize}

\subsubsection{西方马克思主义}
\label{sec:orgb2df05f}
\begin{enumerate}
\item 创始人
\label{sec:orgfac035c}
\begin{itemize}
\item \textbf{卢卡奇} 匈牙利 《历史与阶级意识》
\item \textbf{科尔施} 德国 提出了西方马克思主义概念
\begin{itemize}
\item 晚年关注中国, "马克思主义将在中国复兴"
\item 马克思主义与哲学, 首先提出西马
\end{itemize}
\item \textbf{葛兰西} 意大利
\begin{itemize}
\item 《狱中札记》
\end{itemize}
\end{itemize}

\item 其他
\label{sec:org3abe932}
\begin{itemize}
\item 萨特 存在主义 + 马克思主义
\item 波伏娃 女权主义 + 马克思主义
\item 弗洛伊德 精神分析学嵌入马克思主义
\item 阿尔堵塞 结构主义
\item 生态马克思主义 帝国主义无需扩张导致生态危机
\item 基督教 解放神学
\item \textbf{\textbf{法兰克福学派}} 最重要的学派 德国人为主
\begin{itemize}
\item 第一代 
\begin{itemize}
\item 霍克还莫(启蒙辩证法)
\item 阿朵诺(否定辩证法)
\item 马尔库赛(\emph{单向度的人}, 对发达资本主义批判)
\end{itemize}
\item 第二代 哈被马思
\item 第三代 或奈特
\end{itemize}
\end{itemize}
\end{enumerate}

\subsubsection{生平}
\label{sec:org7ee739b}
\begin{quote}
离经叛道
\end{quote}

\begin{itemize}
\item 1818年出生于德-特里尔, 中产阶级家庭.
\item 1835年进入波恩大学, 法律专业.
\begin{itemize}
\item 1945年分德, 波恩为西德首都.
\end{itemize}
\item 1836年转入柏林大学, 法律专业.
\item 1841年3月毕业
\item 同年5月博士毕业于耶那大学
\begin{itemize}
\item 靠关系
\item 靠论文
\end{itemize}
\item 1842-1843 任 \textbf{莱茵报} 主编
\begin{itemize}
\item 政论性文章, 报纸被查封
\end{itemize}
\item 1843 退居乡下, 小镇克罗兹那赫, 其父友招之入普鲁士国家报做主编, 马克思拒绝
\begin{itemize}
\item 普鲁士政府两次劝马克思回国, 均被拒绝
\begin{itemize}
\item 顾赫尔, 马克思回话:"你总是以自己的尺度测量他人".
\item <!--- --->
\end{itemize}
\end{itemize}
\item 1843, 卢格, 马克思的革命同志, 邀马克思前往政治宽松的法国
\begin{itemize}
\item 办报纸, 写文章, 骂政府, \textbf{德法年鉴} (德国的哲学和法国的革命)
\item \emph{黑格尔法哲学批判导言}, \textbf{列宁认为} 马克思从唯心主义转向唯物主义, 从革命民主主义转为共产主义
\item 1844年被法国驱逐出境, 一同出境的名单有海涅, 但政府受民众要求不驱逐海涅
\end{itemize}
\item 马克思流亡比利时布鲁塞尔, 参加正义者同盟, 改其名为共产主义者同盟, 口号改为全世界共产主义者联合起来, 起草共产党宣言
\item 1848年联合恩格斯回德国办 \emph{新莱茵报}, 被镇压, 再次驱逐
\item 1849年, 马克思恩格斯正式流亡英国
\item 恩格斯回父亲(大资产家)公司赚钱, 马克思理论研究
\begin{itemize}
\item 马克思理论研究贫困潦倒
\item 期间美国纽约论坛报邀马克思写过数篇文章(包括十余篇关于中国的文章), 仍穷困
\end{itemize}
\end{itemize}

\begin{enumerate}
\item 家庭环境: 允许马克思成为三个精英
\label{sec:org8ed4844}
\begin{enumerate}
\item 律界精英
\label{sec:org42f2dad}
\begin{itemize}
\item 父亲是律师, 攻读律师, 人际关系.
\end{itemize}
\item 学界精英
\label{sec:org1114725}
\begin{itemize}
\item 人脉: 柏林大学
\begin{itemize}
\item 甘斯(黑格尔的弟子)的学生.
\item 萨维尼的学生.
\item 青年黑格尔派的领导人, 布鲁诺-鲍威尔, 亦师亦友,
\end{itemize}
指导马克思博士论文 \emph{德谟克利特的自然哲学和伊壁鸠鲁的自然哲学的差别}
\item 博士毕业后, 从教前, 政治变化, 革命教师除职, 不能讲授革命思想了, 放弃学界
\end{itemize}
\item 政界精英
\label{sec:org6804aa3}
\begin{itemize}
\item 夫人贵族出身
\item 大舅内政大臣
\end{itemize}
\end{enumerate}

\item 离经叛道
\label{sec:org087d305}
\begin{itemize}
\item 为劳动人民谋幸福
\item 使马克思恩格斯幸福生活的制度, 建立于对劳动人民的剥削之上
\item 陈独秀, 李大钊等大知识分子, 状况类似
\item 周恩来
\begin{itemize}
\item 出身"落魄的小资产阶级家庭"
\item "我已经背叛了自己的阶级"
\end{itemize}
\item 毛泽东
\begin{itemize}
\item 出身富农家庭
\end{itemize}
\end{itemize}
\end{enumerate}

\subsubsection{著作}
\label{sec:org298a42b}

\begin{quote}
浩如烟海
\end{quote}

\begin{itemize}
\item 1843年, 莱茵报发文 \emph{黑格尔法哲学批判导言}, \emph{论犹太人问题}
\begin{itemize}
\item \textbf{列宁} 认为马克思从唯心主义转向唯物主义, 从革命民主主义转为共产主义
\item 1930年才纠正为马克思在1845年思想成熟
\end{itemize}
\item 1844年 \emph{经济学哲学手稿} (44手稿), 生前未发布
\begin{itemize}
\item 直至1930年才被发现, 学者才发现 \textbf{马克思早年与晚年} 的差别
\item 同年, \emph{神圣家族-对批判的批判所作的批判} (\textbf{合著})
\begin{itemize}
\item 列宁: "三分之二是冗长无聊的文字"
\begin{itemize}
\item 普鲁士不审查过长的书本, 百姓不去读
\end{itemize}
\end{itemize}
\end{itemize}
\item 1845年 \emph{[费尔巴哈的]提纲}, \emph{[德意志意识]形态} (\textbf{合著}), \textbf{成熟代表作}, \textbf{历史唯物主义诞生}
\begin{itemize}
\item 阿尔堵塞:"马克思的人生有一次断裂, 提纲是划过夜空的闪电"
\end{itemize}
\item 1847年 \emph{哲学的贫困} 第一部法语
\begin{itemize}
\item 批判 \emph{贫困的哲学}
\end{itemize}
\item 1848年 \emph{[共产党]宣言} \textbf{第三部合著}, 共三部
\item 1849年 \emph{雇佣劳动与资本} (\textbf{经济})
\item 1852年 \emph{路易波拿巴的雾月十八日} (\textbf{政治})
\item 1859年 \emph{政治经济批判(一)} (仅一册)
\item 1867年 \emph{资本论-政治经济批判(一)} (本人仅出版一卷)
\item 1871年 \emph{[法兰西]内战} (讲巴黎公社) 
\begin{itemize}
\item \textbf{走入老年}
\end{itemize}
\item 1875年 \emph{哥达纲领批判}
\end{itemize}

\begin{enumerate}
\item 手稿存于
\label{sec:org052d874}
\begin{itemize}
\item 德国马琳基金会
\item 
\end{itemize}
全手稿: 2031年MEGA\textsuperscript{2}
\end{enumerate}
\section{第一讲 "新唯物主义" 西方哲学史的一场变革}
\label{sec:org674f061}
\subsection{西方哲学-理念论}
\label{sec:orga035737}

\begin{quote}
西方哲学: 两个世界, 中国哲学: 一个人生
\end{quote}

\subsubsection{两个世界}
\label{sec:orgcd74811}
\begin{itemize}
\item 生活在: 感性世界/现实世界/形而下的世界
\item 真理存在: 超感性世界/理念世界/形而上世界
\end{itemize}
\emph{感性世界的真理存在于理念世界, 感性世界分有理念世界}
\begin{itemize}
\item \textbf{思辩哲学}
\item \textbf{宗教信仰}, 虔诚
\end{itemize}

\subsubsection{中国: 一个人生}
\label{sec:org96775e5}
\emph{形而上的西方哲学传入中国后, 改化为形而下}
\begin{itemize}
\item "两个世界"是统一的
\begin{itemize}
\item "道器不分", "体用不二"
\item "大道不离人伦日用"
\end{itemize}
\item \textbf{不讨论形而上}, "未知生, 焉知死"
\item \textbf{人生哲学/实用哲学}
\item \textbf{宗教实用}, 利用
\end{itemize}

\subsubsection{原因分析}
\label{sec:org10904ed}
\begin{itemize}
\item 雅思贝尔思: 公元前500年->轴心时代
\begin{itemize}
\item 世界各处出现哲学家, 后人按照他们设定的方向发展. (存在主义哲学)
\end{itemize}
\item 中国哲学产生于乱->实用
\item 西方哲学产生于闲->信仰
\end{itemize}

\subsubsection{理念论 Idealism}
\label{sec:org70e622b}
\begin{itemize}
\item 判断唯心主义: 是否划分两个世界
\end{itemize}

\subsection{西方哲学史脉络}
\label{sec:org5f6bbf3}
\subsubsection{柏拉图}
\label{sec:org2b096e2}
合乎理念即是此物
\subsubsection{基督教哲学}
\label{sec:orgd2ec4c3}
\begin{itemize}
\item 感性化, 把理念替换为上帝, 用宗教替代传统思辩哲学, 上帝的理念支配现实世界
\item 上帝->神圣的
\emph{西方哲学主义的信仰一定是神圣的}
\begin{enumerate}
\item 无限性, 无尽名
\begin{itemize}
\item "被描述的时候, 即被否定了"
\end{itemize}
\item 超越性
\begin{itemize}
\item 超越现实世界, 超越人
\item 不能理解, 只能信仰
\end{itemize}
\item 普遍性
\begin{itemize}
\item 现西方"普适价值体系", 已被落实为制度
\item 西方将文化强行外传, 霸权
\item \emph{中国自我中心, 而不向外渗透}
\end{itemize}
\end{enumerate}
\end{itemize}

\subsubsection{笛卡尔}
\label{sec:org0a620e5}
\begin{itemize}
\item 近代哲学, \textbf{人的地位发展}, 主体性转向, 认识论转向, 考量人能否认识理念, 能否把握真理, 而非只看上帝
\item 马克思对哲学的定义: \textbf{哲学是时代在精神上的精华}, 哲学反映时代的命脉, 让这个时代在精神上升华
\begin{itemize}
\item 分析马哲的时代意义
\item 哲学反映了时代
\begin{itemize}
\item 马克思
\begin{itemize}
\item 
\end{itemize}
\item 毛泽东
\begin{itemize}
\item 解放: 新民主主义论
\item 文革: 无产阶级继续革命论
\end{itemize}
\end{itemize}
\end{itemize}
\end{itemize}

\begin{quote}
我思故我在 \\
I think, therefore I am.
\end{quote}
\begin{itemize}
\item 我思哲学
\item I am: 纯存在, 理念的存在
\begin{itemize}
\item 黑格尔: 纯存在, 无规定性的规定性
\end{itemize}
\end{itemize}

\subsubsection{德国古典哲学}
\label{sec:org92b1183}
\begin{quote}
西方哲学的集大成者
\end{quote}
\begin{enumerate}
\item 康德
\label{sec:org664b8f6}
\begin{itemize}
\item 不可知论
\end{itemize}
\begin{enumerate}
\item 两个世界: 现象/经验世界 <-> 本体/超验/先验世界
\label{sec:orgf2d3685}
\begin{itemize}
\item 经验/现象区分
\begin{itemize}
\item 经验世界: 从主体出发, 体验到的世界
\item 现象世界: 从客体出发, 显现出的世界
\end{itemize}
\item 本体世界, 人的感性知性理性均无法认识
\item 批判理性, 为知识划定边界, 从而为信仰留下空间
\end{itemize}
\item 本体世界: 物自体
\label{sec:org7f57d85}
\begin{itemize}
\item 意志自由
\item 灵魂不朽
\item 上帝存有
\end{itemize}
\item 马克思: 康德哲学是法国革命的德国理论, 体现出了德资产阶级的软弱
\label{sec:org6338dae}
\begin{itemize}
\item 革命理念: 上帝是否存有, 此事是本体世界的, 是不可知不可理解的
\begin{itemize}
\item 海涅: 康德砍下了自然神论, 上帝的头颅, 置天平上, 在对侧放上了正确的砝码
\end{itemize}
\end{itemize}
\item 道德要在本体世界找到, 不要与经验世界, 与世俗牵连
\label{sec:org9064b97}
\begin{itemize}
\item "只有可以成为普遍法则的准则, 才是符合道德的", 可接受全世界的人都这么做的事, 是道德
\begin{itemize}
\item 不能撒谎
\item 不能自杀
\item 互相帮助
\item 努力发展自己的才能
\end{itemize}
\item 道德是定言命令式, "应当"
\item 不是假言命令式, "如果"
\end{itemize}
\item 三大批判
\label{sec:org4855145}
\begin{enumerate}
\item 纯粹理性批判: 理论理性低于实践理性, 科学知识应该让位给宗教信仰
\item 实践理性批判: 人在绝对服从道德律令的情况下, 不应该只是去寻找快乐, 而应该去寻找上帝赐予人们的幸福
\item 判断力批判: 寻求两个分割的世界的沟通, 认为自由的道德律令要在感性的现实世界实现出来, 其中介是反思判断力
\end{enumerate}
\end{enumerate}
\item 黑格尔
\label{sec:org552c4ad}

\begin{quote}
费尔巴哈: "黑格尔哲学是宗教哲学的最后支柱"
\end{quote}

核心: \textbf{绝对精神(上帝)}, 但不寄宿于"太抽象"的本体世界
\begin{itemize}
\item 抽象的绝对精神, \textbf{易化到现实世界, 经历现实世界}
\item 再回归到内容丰富的绝对精神
\end{itemize}
\end{enumerate}

\subsection{马克思超越理念论}
\label{sec:org6fdc433}
\subsubsection{何为超越}
\label{sec:org4b9359b}
不是简单的颠倒, 否定, 停留在旧有的结构; 而是改变它的形态
\begin{itemize}
\item 身是菩提树, 心如明镜台, 时时勤拂拭, 勿使惹尘埃.
\item 菩提本无树, 明镜亦非台, 本来无一物, 何处惹尘埃?
\end{itemize}
\subsubsection{物质本体论}
\label{sec:orgded702c}
\begin{itemize}
\item 存在的问题: 马克思不支持; 没有超出理念论
\item 也称万有(诸存在)/存在论: 研究世间诸存在的问题
\end{itemize}
\begin{enumerate}
\item 客观实在性
\label{sec:org6c6a8a2}
\item 永恒不灭性
\label{sec:orgc3e1f98}
\item 形而上的理念性, 没有超出理念论
\label{sec:org1892014}
现实社会的物体分有了理念, 具备着物质性
\end{enumerate}
\subsubsection{马-恩的第一个差异\hfill{}\textsc{马:恩的第一处差异}}
\label{sec:org88eda18}
恩格斯停留在传统哲学理论: 物质本体, 而马克思提出了超越理念论
\subsubsection{马克思: 实践本体论}
\label{sec:org370beab}
\begin{enumerate}
\item \emph{关于费尔巴哈的提纲}
\label{sec:org70c2563}
\begin{quote}
关于费尔巴哈的提纲
\end{quote}
\begin{enumerate}
\item 费尔巴哈的唯物主义 (1. 关于费尔巴哈, 一)
\label{sec:org854e630}
\begin{itemize}
\item 人本学, 感性的人, 把人与人的关系视作哲学的基础
\item 人在对象中认识自己, 主体的性质赋予对象, 对象映射主体的性质
\item 找无限的对象反映人无限的思想: 上帝
\begin{itemize}
\item 为了承载无限的思想, 我们 \textbf{创造了上帝}
\end{itemize}
\item 对象: 例
\begin{itemize}
\item alphaGo是一群人类主体对围棋等的认知, 对象化的产物, 不是所谓机器
\end{itemize}
\end{itemize}

\item 马克思: 实践本体论(实践哲学) (1. 关于费尔巴哈, 六)
\label{sec:orgf4b7677}
\begin{quote}
提纲: 费尔巴哈把宗教的本质归结于人的本质。但是,人的本质不是单个人所固有的抽象物,在其现实性上,它是一切社会关系的总和。\\
提纲: 全部社会生活在本质上是实践的。
\end{quote}
\begin{minted}[]{text}
人(主体) <--建构-- 实践(无前提) --建构--> 世界(对象)
\end{minted}
\begin{itemize}
\item 人也是在实践中被塑造出来
\end{itemize}
\begin{enumerate}
\item 主要问题
\label{sec:orgc4ba9f8}
人实践之外的自然是实践的产物吗?
\begin{itemize}
\item 一种意义上说, 人类发现自然, 则实践建构自然
\end{itemize}
\end{enumerate}
\item "人的思维是否具有客观的(对象性的)真理性,这不是一个理论的问题,而是一个实践的问题" (1. 关于费尔巴哈, 二)
\label{sec:org920857a}
是"一定"的, 不是绝对的, 是生成的过程
\begin{enumerate}
\item "实践是检验真理的唯一标准"
\label{sec:org9b06f93}
\begin{itemize}
\item 政治上, 从两个凡是, 有进步意义
\item 哲学上, 站不住脚, 实践是根本性的检验, 但不是唯一的标准
\end{itemize}
\item 实践关系会改变客观的真理
\label{sec:org07f2a28}
例:
\begin{itemize}
\item \textbf{自由} 是好东西吗?
\begin{itemize}
\item 马克思: 不是理论问题, 是实践问题, 要在实践中证明
\item 实践在 \textbf{中世纪}, 自由不是好东西, \textbf{依附关系} 才是
\item 实践展开至现代, 自由就是好东西了
\begin{itemize}
\item 如今是形式上的自由, 但已经有进步
\end{itemize}
\end{itemize}
\item ***是中国的领土吗
\begin{itemize}
\item 依不同时代的实践而定
\end{itemize}
\item 地心说
\begin{itemize}
\item 农耕文明时, 我们的实践没有超出地球, 地球就是中心
\item 实践发展, 我们观察宇宙时, 太阳才是中心
\end{itemize}
\end{itemize}
\end{enumerate}
\end{enumerate}
\end{enumerate}
\subsection{唯物史观}
\label{sec:org436d138}
\begin{quote}
恩格斯生前用"唯物史观"指代他的研究
\end{quote}
\subsubsection{名谓之争}
\label{sec:orge29e6bb}
\begin{enumerate}
\item 辩证唯物主义
\label{sec:orgddd768b}
\begin{itemize}
\item 狄慈根提出, \textbf{而非马克思}
\item 基础, 涉及到 \textbf{物质本体论}
\item 支持 \textbf{物质本体论} 的人
\end{itemize}
\item 历史唯物主义
\label{sec:org477b868}
\begin{itemize}
\item 斯大林时期
\item \textbf{推广论}: 历史唯物主义是辩证唯物主义在政治历史层面的推广, 应用
\begin{itemize}
\item 辩证是基础
\end{itemize}
\end{itemize}
\item 实践唯物主义\hfill{}\textsc{德意志意识形态}
\label{sec:orge29cf5f}
\begin{itemize}
\item 马克思提出
\item "而且对实践的唯物主义者 \textbf{即共产主义者} 来说\ldots{}" --\emph{德意志意识形态}
\item 支持 \textbf{实践本体论} 的人
\end{itemize}
\item 调和论
\label{sec:org458d4a9}
\end{enumerate}
\subsubsection{"实践"的深化}
\label{sec:org6796421}
\begin{enumerate}
\item 生产\hfill{}\textsc{德意志意识形态}
\label{sec:org00855a1}
\begin{itemize}
\item 包括物质的生产和人的生产
\item 生产实践决定人
\end{itemize}
\begin{quote}
它是这些个人的一定的活动方式, 是他们表现自己生命的一定方式, 他们的一定的生活方式.\\
个人怎样表现自己的生命, 他们自己就是怎样. 因此, 他们是什么样的, 这同他们的生产是一致的——既和他们生产什么一致, 又和他们怎样生产一致.\\
因而, 个人是什么样的, 这取决于他们进行生产的物质条件.
\end{quote}
\item 交往
\label{sec:orge680c8d}
\end{enumerate}
\subsubsection{唯物史观的线索}
\label{sec:orgb52b406}
个人一定的生产产生了一定的社会, 进而产生了政治
\begin{itemize}
\item "民国后还有人问现在谁是皇帝"
\begin{itemize}
\item 要变革社会, 先变革生产关系
\end{itemize}
\item 李约瑟难题
\begin{itemize}
\item 解决问题需要深入到物质生产生活领域
\end{itemize}
\item 生产力决定生产关系 \emph{批判}, 经济基础决定上层建筑 \emph{批判}, 社会存在决定社会意识 \emph{形态}
\item 人民群众是历史的创造者
\begin{itemize}
\item 二战只是希特勒的错? -> 个人英雄主义 (唯心史观)
\item 如何定义 \textbf{人民}
\begin{itemize}
\item 政治场域: 人民是褒义词
\item 历史场域: 人民是处在生产关系中的, 受到极端环境的影响, 会变成极端的人民
\end{itemize}
\item 凡尔塞条约压榨德国, 极端的人民支持纳粹党, 挑起二战
\item 极端的人民支持Trump
\item 台湾和大陆的不合形式, 极端的人民支持台独
\end{itemize}
\end{itemize}
\subsubsection{批判唯心史观}
\label{sec:org8fba177}
\section{第二讲 重拾辩证法的革命内核}
\label{sec:org9714e38}
\subsection{日常生活中的"辩证"}
\label{sec:org9843cea}
\begin{quote}
"凡事都有两面性"
\end{quote}
乡愿哲学
\begin{itemize}
\item 静态的把事物分为两面
\end{itemize}
\subsection{一. 矛盾的辩证法}
\label{sec:org26a8666}
\subsubsection{1. 词源: dialectic <- dialogue}
\label{sec:org3fc945b}
\begin{itemize}
\item 对话, 辩论, 雄辩
\item 古希腊哲学家发现万物都可以一正一反判断
\item 一正一反两面的 \textbf{矛盾}
\end{itemize}
\subsubsection{2. 康德}
\label{sec:orgb73fa40}
\begin{quote}
康德每句话很清楚明确容易懂, 但整体不知道在说什么
\end{quote}
\begin{itemize}
\item 称"自相对立的两个命题同时存在"的现象为二律背反
\end{itemize}
\begin{enumerate}
\item 康德证明了四组二律背反
\label{sec:org13fbda5}
\begin{enumerate}
\item 世界是无限的/有限的
\begin{itemize}
\item 正反观点都是错的
\end{itemize}
\item 世界是可分的/不可分
\begin{itemize}
\item 正反观点都是错的
\end{itemize}
\item 人在世界上是自由的/是必然的结果(因果律)
\begin{itemize}
\item 存在于理念世界/现实世界
\end{itemize}
\item 历史有绝对的起点终点/历史是无穷的, 不存在始终
\begin{itemize}
\item 存在于现实世界/理念世界
\end{itemize}
\end{enumerate}
\end{enumerate}
\subsubsection{3. 黑格尔 矛盾辩证法}
\label{sec:org31bd50a}
\begin{quote}
黑格尔每一句话都很晦涩, 但结构性很强, 整体能读懂
\end{quote}
\begin{enumerate}
\item 黑格尔定义辩证法
\label{sec:orgcb0980e}
一个存在物, 要扬弃自身(变为反面), 走向它的对立面(形成一个新的存在物), 
再扬弃自身\ldots{}, 但否定之否定不是肯定本身, 而是螺旋升天
\begin{enumerate}
\item "扬弃": Aufhebung
\label{sec:orgaa5518c}
\begin{itemize}
\item 废除
\item 举起, 提高
\item 保留, 保存
\item 在德语中, 三个意思是完全独立, 依语境调用的
\end{itemize}

\begin{quote}
百度百科: \\
对原有事物既要抛弃其消极因素, 又要保留, 发扬其积极因素
\end{quote}

\begin{quote}
Max-\emph{共产党宣言}: 
共产党人可以把自己的理论概括为一句话: 消灭(Aufhebung)私有制
\end{quote}

马克思解释说
\begin{itemize}
\item 一方面, 任何革命都会废除一定的财产, "不是共产主义革命的专利"
\item 不是废除一般意义上的财产, 而是废除资产阶级生产资料这一级的财产, 使生产压迫不再存在
\end{itemize}

但是黑格尔此处是哲学的用法
\begin{itemize}
\item 废除自身到达对立面的时候, 我们在对立面也能看到自身作为衬托而存在
\item 没有原先学渣, 就没有如今学霸
\item "正如Aufhebung, 废除的同时, 也保留着旧东西"
\end{itemize}
\end{enumerate}
\end{enumerate}
\subsection{二. 革命辩证法}
\label{sec:orgfa7d62d}
\subsubsection{黑格尔的辩证法中包含革命部分(马克思说)}
\label{sec:org9ee3127}
\begin{itemize}
\item 扬弃后扬弃后扬弃, 旧的时代不断消逝, 新的历史不断创生, 这反映了历史进步的原则
\item 发展到范围的极限, 成为一种新的存在
\begin{itemize}
\item 如自由资本主义发展到垄断资本主义到帝国主义, 到达极限, 而发展为共产主义
\end{itemize}
\item 黑格尔第一次将历史视作不断发展不断进步的过程
\item 黑格尔是比较隐晦的表达, 避风头, 小心谨慎
\item 黑格尔的矛盾/革命辩证法是同一个东西
\item 黑格尔说历史需要从中国说起
\begin{itemize}
\item 中国是一个实体, 不需要依托而存在, 成为其它存在的依托
\item 中国是最古老的国家亦是最新的国家
\end{itemize}
\begin{quote}
马克思: 中国是木乃伊
\end{quote}
\item 黑格尔说世界由东向西发展 
\begin{itemize}
\item 现代日尔曼(理想的君主立宪, 兼具普遍自由和个体自由, 是世界最高峰, 代表欧洲)
<-古罗马(普遍自由而压制个体自由)
<-古希腊(个体自由)
<-古中国(仅君主自由)
\item 于是有人说黑格尔奴颜婢膝, 历史终结论者
\item 而历史会继续向前
\begin{itemize}
\item "当欧罗巴(Europe)成为历史杂物库的时候, 亚美利加(America)将成为新的明日之国,
世界历史的新的中心, 到那里, 世界历史将继续新的使命"
\end{itemize}
\item 为什么是按照空间按照地域尺度?
\item 为什么以自由为量度?
\begin{itemize}
\item 黑格尔仍是唯心主义, 理念主义
\end{itemize}
\item 为什么不去写美国?
\begin{itemize}
\item "历史哲学是已经发生的事情, 亚美利加是明日之国"
\end{itemize}
\end{itemize}
\end{itemize}
\subsubsection{马克思(恩格斯)的辩证法: 批判的革命的辩证法}
\label{sec:org94c8fba}
\begin{quote}
辩证法, 在其合理形态上, 引起资产阶级及其空论主义的代言人的恼怒和恐怖, 
因为辩证法在对现存事物的肯定的理解中同时包含对现存事物的否定的理解, 即对现存事物的必然灭亡的理解;
辩证法对每一种既成的形式都是从不断的运动中, 因而也是从它的暂时性方面去理解;
辩证法不崇拜任何东西, 按其本质来说, 它是批判的和革命的
\end{quote}
\begin{enumerate}
\item 对历史是以展开的视角看
\label{sec:orgbf6d819}
\begin{itemize}
\item 如, 资本主义, 打破了封建主义, 发展了世界, 是革命性的
但在资本主义发展的过程中, 它压迫劳动力, 它必然灭亡, 扬弃自身, 被一个新的环节取代
\item 如, 宗教, 有一定的地位, 有一定的意义
但在发展的过程中, 它将扬弃自身, 被一个新的历史环节取代
\end{itemize}
\item 恩格斯对辩证法的理解
\label{sec:orgf6fa2a5}
恩格斯解释黑格尔一句话
"凡是合乎理性的东西都是现实的, 凡是现实的东西都是合乎理性的"
\begin{itemize}
\item "存在即合理" 是错误的表述
\item 这句话中仍包含革命性
\begin{itemize}
\item 现实(wirklich)不是现存, 现实是在历史展开过程中体现为必然性的东西, 现存是当前存在
\item \textbf{在未来}, 合理的东西在未来将体现为必然性, 而实现的东西一定得是合理的
反之, \textbf{在未来}, 不合理的东西一定会消失, 合理的东西, 在未来展开过程中将体现为必然性
\end{itemize}
\item 恩格斯说黑格尔是无意识说出来的, 没意识到此话的革命性
\end{itemize}
\begin{enumerate}
\item 追问一句: 什么是合理?
\label{sec:orgaf981b0}
\begin{itemize}
\item 暴露出黑格尔辩证法唯心主义的一面: 合乎绝对精神(自由精神)的就是合理的
\item 有生有灭, 循环迭代的是现实世界, 绝对精神是不会迭代的
\item 马克思, 恩格斯, 毛泽东则 \textbf{超越} 这一点
\begin{itemize}
\item 一切事物都是有生有灭的, 都接受辩证法的审判
\item 不"乞灵"于形而上的世界
\end{itemize}
\end{itemize}
\item 恩格斯说我们能推出一个新的命题
\label{sec:org34cf19f}
\textbf{凡是现存的就一定要灭亡}
现存的东西, 在日后将丧失自己的现实性, 进而灭亡
\end{enumerate}
\item 老子: 祸兮, 福之所倚, 福兮, 祸之所伏
\label{sec:org76f8042}
\item 佛教: 诸行无常
\label{sec:org0d17bde}
\end{enumerate}
\subsubsection{毛泽东的革命辩证法}
\label{sec:orgb2e7bd4}
\begin{quote}
马克思主义的理论家, 战略家, 革命家
(邓小平: 政治家, 外交家, 军事家, 革命家)
\end{quote}
\begin{itemize}
\item 理论家: 革命思想的输出
两个标准
\begin{itemize}
\item 有没有外国革命竖起这个思想的大旗
\item 国外研究思想史的人, 有没有研究这个思想
\end{itemize}
\end{itemize}
\begin{enumerate}
\item 把辩证法展开为三个环节
\label{sec:orgbeb98c2}
\begin{enumerate}
\item 人的必然灭亡
\label{sec:orgc030fbc}
\begin{quote}
讲辩证法, 却不讲死亡, 这不好, 是形而上学
\end{quote}
\item 社会形态的必然灭亡
\label{sec:org119c307}
\begin{quote}
社会主义作为一种历史现象, 也必然灭亡
\end{quote}
资本主义 -> 社会主义 -> 共产主义
\begin{itemize}
\item 到共产主义也是有革命的, 生产力和生产关系不对的时候, 人民也会发起革命
\end{itemize}
\item 政党和国家的必然灭亡\hfill{}\textsc{论人民民主专政}
\label{sec:orgec1c862}
\begin{quote}
作为阶级斗争的工具, 政党和国家也将死亡
\end{quote}
\begin{enumerate}
\item 在消灭阶级的意义上中国必然灭亡
\label{sec:orga5ad331}
\item 在现实意义, 党的自我革命
\label{sec:org6b51ca3}
\begin{itemize}
\item 扬弃自身落后的东西, 促使新的合乎时代的东西生成, 促进机体新陈代谢
\item 今日共产党非昨日--习近平
\end{itemize}
\end{enumerate}
\end{enumerate}
\end{enumerate}
\subsubsection{教科书体系的辩证法}
\label{sec:orgdfe416a}
\begin{itemize}
\item 矛盾的观点, 联系的观点, 发展的观点
\item 矛盾, 发展同马克思, 联系的观点源于斯大林
\end{itemize}
\begin{enumerate}
\item 斯大林: 辩证法的三个特征 <-> 形而上学的三个特征\hfill{}\textsc{论辩证唯物主义和历史唯物主义}
\label{sec:org72272cb}
\begin{itemize}
\item 转译形而上学: meta-physik -> anti-dialectic
\begin{itemize}
\item 形而上的东西不会灭亡, 违背辩证法一切事物有生有灭, 则形而上即反辩证法
\end{itemize}
\end{itemize}
\begin{enumerate}
\item 联系的 <-> 孤立的
\label{sec:orga8a446f}
\begin{itemize}
\item 普遍联系的观点: 相对立的两物正是联系起来的
\begin{itemize}
\item 细胞在新陈代谢中, 是这个细胞, 又不是它
\item 生命本身蕴含着死亡的种子
\end{itemize}
\end{itemize}
\item 运动发展的 <-> 静止的
\label{sec:orgc23a9fc}
\item 从量变到质变的 <-> 简单增长的
\label{sec:org7a60134}
\item 三者其实是同一个辩证法
\label{sec:orgbbff875}
\begin{itemize}
\item 核心的载体是革命的辩证法, 是历史演化的过程
\end{itemize}
\end{enumerate}
\end{enumerate}
\end{document}